% !TeX root = ../main.tex

\chapter{总结和展望}

本文关注的是云闪存系统中的多租户性能隔离问题。
特别地,本文试图减少由共享同一块物理SSD的多个租户之间的读写干扰而导致的尾延迟问题。
为解决这一问题,本文首先提出用SLA曲线来描述用户对于带宽的需求,这既贴近实际的SSD性能,也充分体现了SSD读写不对等的特性。

基于以上定义,本文从两方面解决云闪存中的多租户性能隔离问题。
第一种方法是通过分离的读写时间窗口构建无读写干扰的SSD阵列,其目的是彻底避免租户间的读写干扰。
通过对SSD阵列构建方式和时间窗口大小的选择,该SSD阵列可以在保证较大读写吞吐的情况下保持接近于纯读的读延迟。
第二种方法是基于SLA曲线对存储资源进行分配,其目的是尽可能减轻租户间的读写干扰。
通过在数据分布时确保租户的SLA需求不会超过SSD的限制,并在运行时动态根据SLA曲线进行IO请求调度,本系统可以利用尽可能少的资源满足租户的SLA需求。
实验显示,综合使用上述两种方法,可能将云闪存系统整体的资源使用量减少一半。

在此指出本工作的如下缺陷,也是对未来工作的展望:

首先,本系统搭建的无读写干扰的SSD阵列尽管具有良好的尾延迟,但是其利用冗余导致整体的资源利用率降低。
RAID-4使用的冗余盘较少,具有解决该问题的潜力,但由于\autoref{chap:eval}提及的读写放大问题,其能够提供的读写吞吐不理想。
如何在减少冗余的同时还能提供良好的读写吞吐是一个可能的优化方向。
% RAID-1和RAID-4的读放大问题未解决:更丰富的资源调度方式。考虑吞吐、request size、address pattern,批处理任务?

此外,本文的资源分配基于SLA曲线,但是并未解决如何获取SLA曲线的问题。
从租户的角度来讲,由其输入SLA曲线并不现实,而且也未必准确。
因此,通过Profile来获取租户的SLA曲线是一个可能的解决方法\cite{delimitrou2014quasar};
从硬件的角度来讲,在不同的请求大小、访问地址分布的情况下,SLA曲线可能会有所不同。
在这样变化的环境下,如何动态地确保SSD和租户始终得到SLA保证也是一个可以深入研究的方向。

