% !TeX root = ../main.tex

\ustcsetup{
  keywords = {
    存储系统, 闪存, 多租户, 资源池化, 服务质量
  },
  keywords* = {
    Computer Storage System, Flash Storage, Multi-Tenancy, Resource Disaggregation, Quality of Service
  },
}

\begin{abstract}
  由于其低延迟、高吞吐的特点,\textit{闪存}在云系统中得到了越来越广泛的应用。
  同时,为了解决数据中心资源利用率低的问题,科研人员提出了\textit{资源池化},通过将计算与存储等解耦合,使资源能更高效地在租户之间共享。
  在资源池化的存储系统中,多租户间\textit{性能隔离}、互不干扰是一个重要需求。
  然而,在闪存中,一个租户的写请求可能触发闪存内部的垃圾回收等维护行为,这些行为会严重影响其它租户的访问性能,从而使租户间的性能隔离难以实现。
  由于以上的\textit{读写干扰}问题,在对延迟要求较高的云服务中,闪存大多被租户独占。
  
  本文通过减少上述读写干扰问题,构建了一个具有多租户性能隔离特性的闪存资源池化系统。
  \textbf{1)}本文通过分配隔离的纯读和纯写时间窗口,使闪存在纯读窗口中获得不受干扰的读延迟,在纯写窗口中通过批量化得到最大的写带宽;
  \textbf{2)}本文利用具有以上特征的闪存搭建具有不同冗余特征的闪存阵列,并通过协调各块闪存中的时间窗口,使得任一时刻对任何数据的访问都能获得不受干扰的读延迟;
  \textbf{3)}在此基础上,本文基于\textit{用户服务曲线}和装箱问题,提出了一个闪存资源池化系统中不同闪存阵列选择和资源分配的贪心算法。
  试验结果表明本文提出的系统具有良好的性能隔离效果和高效的资源利用率。
  \JF{加实验结果}
\end{abstract}

\begin{abstract*}
  Thanks to its low latency and high throughput, NAND-based flash storage is increasingly gaining popularity in cloud datacenters.
  Meanwhile, researchers have proposed \textit{resource disaggregation} to deal with the low resource utilization in datacenters.
  In such disaggregated storage systems, \textit{multi-tenant performance isolation} is a vital requirement.
  However, on a flash disk, write requests from a certain tenant may trigger the disk's internal maintenance operations,
  which in turn will impact access performance from other tenants, making it hard to achieve performance isolation.
  Because of this \textit{read-write interference} problem, each flash disk are usually assigned to only one user.

  This thesis presents a disaggregated flash storage system with multi-tenant isolation by minimizing read-write interference.
  \textbf{1)} This work proposes to assign isolated pure-read and pure-write time windows on flash disks,
  thus achieving un-interferenced read latency during read windows
  and maximized write bandwidth during write windows through batching.
  \textbf{2)} This work builds fault-tolerant flash arrays based on these read-write isolated disks which provide
  un-interferenced read latency at \textit{any} time by orchaestrating time windows on different disks.
  \textbf{3)} Furthurmore, this work designs a greedy algorithm for the resource allocation problem on disaggregated flash storage systems
  based on \textit{SLA Curve} and the bin packing problem.
  Experiments show that the proposed system achieves good performance isolation and high resource utilization.
  \JF{Experiment results}
\end{abstract*}
